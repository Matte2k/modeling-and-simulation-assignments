
\documentclass[11pt,a4paper,oneside]{article}
\usepackage[T1]{fontenc} 
\usepackage[utf8]{inputenc}
\usepackage[main=english]{babel}
\usepackage{graphicx} % 1pt = 0.035146cm
\usepackage[justification=default]{subfig} %Manage sub-figures 
\usepackage[update]{epstopdf}
\usepackage[labelfont=bf]{caption}
\usepackage{titlesec} % Allows customization of titles
\usepackage{booktabs}
\usepackage{gensymb}
\usepackage{color}
\usepackage[textwidth = 450pt,top = 80pt, bottom = 60pt]{geometry}
\usepackage{soul}
\newcommand{\hlc}[2][yellow]{{\sethlcolor{#1}\hl{#2}}}
\usepackage[inline]{enumitem}
\usepackage[symbol]{footmisc}

\renewcommand{\thefootnote}{\fnsymbol{footnote}}

%--------------------------------------------------------------------------------
%       MATH PACKAGES
%--------------------------------------------------------------------------------
\usepackage{amsmath}
%\usepackage{mathtools}
\usepackage[leqno,fleqn,intlimits]{empheq}
\usepackage{bm}
%\usepackage{amssymb}
\usepackage{empheq}
\usepackage[inline]{enumitem}
%--------------------------------------------------------------------------------
%       MATLAB CODE
%--------------------------------------------------------------------------------
\usepackage{mcode}
%--------------------------------------------------------------------------------
%       MY COMMANDS
%--------------------------------------------------------------------------------
\renewcommand{\vec}[1]{\mathbf{#1}}
\newcommand{\tr}{\textcolor{red}}
\newcommand{\mathbi}[1]{\bm{\textbf{\em #1}}}
%--------------------------------------------------------------------------------
%       BIBLIOGRAPHY PACKAGES
%--------------------------------------------------------------------------------
% \usepackage{csquotes}
% \usepackage[sorting=nyt,%
% sortcites=true,%
% bibencoding=ascii,%
% autopunct=true,%
% hyperref=true,%
% language=auto,%
% %backref=true,%
% url=false,%
% maxcitenames=10,%
% minbibnames = 3,%
% maxbibnames=3,%
% giveninits, 
% natbib = false,
% isbn=false,%
% backend=biber]{biblatex}
% \addbibresource{bibliograhy_assignment.tex}
%--------------------------------------------------------------------------------
%       MISCELLANEA
%--------------------------------------------------------------------------------
\usepackage[]{hyperref}
\usepackage{cleveref}
%%% CREF setup
\crefname{equation}{Eq.}{Eqs.}
\crefname{table}{Table}{Tables}
\crefname{figure}{Fig.}{Figs.}

%--------------------------------------------------------------------------------
%       TITLE SECTION
%--------------------------------------------------------------------------------
\newcommand\headlinecolor{\normalcolor}

\makeatletter
\renewcommand*\maketitle{%
    \begingroup
    \centering
    \fontsize{15}{15}% 72pt on 80pt leading
    \selectfont
    \headlinecolor
    \@title\\
    \vspace{5mm}
    \@author
    \par
    \vskip1in
    \endgroup
    \vspace{-22mm}
}
\makeatother


\title{MSAS -- Assignment \#2: Modeling} % Article title
\author{\large \tr{Student Name}, \tr{123456}}
\date{}

%--------------------------------------------------------------------------------
% HEADING packages
\usepackage{fancyhdr} % Headers and footers control
\setlength{\headheight}{15.2pt}
\pagestyle{fancyplain} % Defines a new header for all pages (absolutely all pages, use fancy to exclude title-page and chapters, if book class is used) 
\fancyhf{} % clears the header and footer, otherwise the elements of the default "plain" page style will appear
%
\lhead{\tr{Student Name}, MSAS -- Assignment \#2}
\rhead{\vspace{-0.5cm}\includegraphics[width=0.3\textwidth]{newlogo.eps}}
\lfoot{AY 2023-24 -- Prof.\ F.\ Topputo; TA: C.\ Balossi, S.\ Borgia}
\rfoot{\thepage}
%--------------------------------------------------------------------------------
%       BEGIN DOCUMENT
%--------------------------------------------------------------------------------
\begin{document}

\maketitle

\thispagestyle{fancy}
%--------------------------------------------------------------------------------
%       MODELING PARADIGMS, BASIC SYSTEMS
%--------------------------------------------------------------------------------


\section*{Exercise 1}
The rocket engine in Figure \ref{fig:therm} is fired in laboratory conditions. With reference to Figure \ref{fig:therm}, the nozzle is made up of an inner lining ($k_1$), an inner layer having specific heat $c_2$ and high conductivity $k_2$, an insulating layer having specific heat $c_4$ and low conductivity $k_4$, and an outer coating ($k_5$). The interface between the conductor and the insulator layers has thermal conductivity $k_3$.

\subsection*{1.1) Part 1: Parameters definition}
Select the materials of which the nozzle is made of\footnote{The interface layer is not made of a physically existing material, though it produces a thermal resistance. For this layer, the value of the thermal resistance $R_3$ can be direcly assumed, so avoiding to choose $k_3$ and $\ell_3$.}, and therefore determine the values of $k_i$ ($i=1,\dots,5$), $c_2$, and $c_4$. Assign also the values of $\ell_i$ (i=1,\dots,5), $L$, and $A$ in Figure \ref{fig:therm}.
\begin{figure}[h!]
\centering
\includegraphics[width=0.8\textwidth]{fig_therm.pdf}
\caption{\label{fig:therm} Real thermal system.}
\end{figure}

\subsection*{1.2) Part 2: Causal modeling}
Derive a physical model and the associated mathematical model using one node per each of the five layers and considering that only the conductor and insulator layers have thermal capacitance. The inner wall temperature, $T_i$, as well as the outer wall temperature, $T_o$, are assigned.
Using the mathematical model, carry out a dynamic simulation \underline{in MATLAB} to show the temperature profiles across the different sections. At initial time, $T_i(t_0)=T_o(t) = 20\ {\rm C^\circ}$. When the rocket is fired, $T_i(t) = 1000\ {\rm C^\circ}$, $t\in[t_1,\,t_f]$, following a ramp profile in $[t_0,\,t_1]$. Integrate the system using $t_1 = 1\ {\rm s}$ and $t_f = 60\ {\rm s}$.

\subsection*{1.3) Part 3: Acausal modeling}
a) Reproduce \underline{in Simscape} the physical model derived in Part 2. Run the simulation from $t_0 = 0\ {\rm s}$ to $t_f = 60\ {\rm s}$ and show the temperature profiles across the different sections. Compare the results with the ones obtained in point 1.2). 
b) Which solver would you choose? Justify the selection based on the knowledge acquired from the first part of the course.
c) Repeat the simulation in Simscape implementing two nodes for the conductor and insulator layers and show the temperature profiles across the different sections.
\medskip

\medskip \hrule \medskip
\rightline{\small(15 points)}

\tr{\textit{Write your answer here}}
%%%%%%%%%% TO BE DELETED (BEGINS) %%%%%%%%%%
\tr{
\begin{itemize}
\item Develop \underline{one Matlab script} for Exercise 1; name the file \mcode{lastname123456_Assign2.m}. If needed, organize the script in sections and use local functions. 
\item Develop \underline{one Simulink model} for Exercise 1; name the file \mcode{lastname123456_Assign2.slx}. 
\item Develop \underline{two Dymola models} for Exercise 2; name the files \mcode{lastname123456_Assign2_Part_1.mo} and  \mcode{lastname123456_Assign2_Part_2.mo}. 
\item \underline{Create a single .zip file} containing the \underline{report in PDF}, the \underline{MATLAB file}, the \underline{Simulink model}, and the \underline{two Dymola models}. The name shall be \mcode{lastname123456_Assign2.zip}.
\item Red text indicates where answers are needed; be sure there is no red stuff in your report.
\item In your answers, \underline{be concise}: to the point.
\item \textbf{Deadline for the submission: Dec 18 2023, 23:59}.
\item \textbf{Load the compressed file to the Homework folder on Webeep}.
\end{itemize}
}
%%%%%%%%%% TO BE DELETED (ENDS) %%%%%%%%%%

\section*{Exercise 2}
The real system of an electric propeller engine is depicted in Figure \ref{fig:propeller}. It is composed by a DC permanent magnet motor which drives a propeller shaft. Between the motor and propeller shaft there is a single stage gear box to regulate the angular speed ratio. Moreover, to avoid overheating of the gear unit, the system is augmented by a cooling system where a fluid exchanges heat with the gear box itself. In Figure \ref{fig:blocks scheme} a functional breakdown structure of the system is shown. 



\begin{figure}[h!]
\centering
\includegraphics[width=0.8\textwidth]{RealSystem_msas.png}
\caption{\label{fig:propeller} Real system.}
\end{figure}


\begin{figure}[h!]
\centering
\includegraphics[width=0.9\textwidth]{BlockScheme_msas.png}
\caption{\label{fig:blocks scheme} Functional block scheme of the system.}
\end{figure}

\subsection*{2.1) Part 1: Propeller Electric Engine}
Considering the real system in \ref{fig:propeller} \textbf{without} the cooling part, you are asked to:
\begin{enumerate}
    \item Extract a physical model highlighting assumptions and simplifications.
    
    \item Reproduce the model in acausal manner in Dymola.
    
    \item According to the block scheme in \ref{fig:blocks scheme}, tune a controller (e.g., a PID controller) such that the motor input voltage remains less than 200 V and the error signal \textbf{Err} is less than 0.1 rad/s after 10 s. 

    \item Study the Gear box temperature and heat flux for a simulation time of $\mathbf{t_f}$ = 120 s (considering only conduction as heat transfer).

    \item Discuss the simulation results and the integration scheme used
\end{enumerate}

\noindent For the simulation part, you shall consider: the DC motor data listed in Table \ref{tab:dc}; the gear box data listed in Table \ref{tab:gear}, with loss parameters in Table \ref{tab:gear2}; a propeller made of \textbf{aluminium} with nominal angular speed $\hat{\omega}$ and a nominal quadratic speed load torque $\hat{T}_{load}$ acting on it (Table \ref{tab:prop}). The reference angular speed signal to be tracked by the propeller is given in Figure \ref{fig:refsignal}. 


\begin{table}[h!]
\centering
\caption{DC motor data}
\begin{tabular}{ |c|c|c| } 
\hline
\textbf{Parameter} & \textbf{Value} & \textbf{Unit}\\
\hline
Coil Resistance & 0.1 & $\Omega$  \\ 
Inductance & 0.01 & H  \\ 
Motor Inertia & 0.001 & kg $m^2$  \\ 
Motor Constant & 0.3 & Nm/A \\ 
\hline
\end{tabular}
\label{tab:dc}
\end{table}

\begin{table}[h!]
    \centering
    \caption{Gear Box data}
\begin{tabular}{ |c|c|c| } 
\hline
\textbf{Parameter} & \textbf{Value} & \textbf{Unit}\\
\hline
Mass & 3 & kg  \\ 
Gear ratio & 2 & [-]  \\ 
Specific heat & 1000 & J/(kg K)   \\ 
Thermal Conductivity & 100 & Wm/K \\ 
\hline
\end{tabular}
\label{tab:gear}
    \end{table}
    
\begin{table}[h!]
    \centering
    \caption{Gear Box Loss Table}
\begin{tabular}{ |c|c|c| } 
\hline
\textbf{Driver angular speed [rad/s]} & \textbf{Mesh efficiency[-]} & \textbf{Bearing friction torque [Nm]}\\
\hline
0 & 0.99 & 0  \\ 
50 & 0.98 & 0.5  \\ 
100 & 0.97 & 1  \\ 
210 & 0.96 & 1.5  \\ 
\hline
\end{tabular}
\label{tab:gear2}
    \end{table}

\begin{table}[h!]
    \centering
    \caption{Propeller data}
\begin{tabular}{ |c|c|c| } 
\hline
\textbf{Parameter} & \textbf{Value} & \textbf{Unit}\\
\hline
Diameter & 0.8 & m  \\ 
Thickness & 0.01 & m  \\ 
$\hat{\omega}$ & 210 & rad/s \\
$\hat{T}_{load}$ & 100 & Nm \\
\hline
\end{tabular}
\label{tab:prop}
    \end{table}



    \begin{figure}[h!]
\centering
\includegraphics[width=0.8\textwidth]{ReferenceSignal.png}
\caption{\label{fig:refsignal} Angular speed reference for the propeller.}
\end{figure}



\subsection*{2.2) Part 2: Cooling System}
After the previous gear unit thermal analysis, now consider the steady-state condition reached by the propeller engine at the end of the simulation to model and simulate a single \textbf{fixed} volume flow rate cooling system (as shown in Figure \ref{fig:propeller}) for the gear unit and considering only \textbf{convection} as heat transfer. In particular, you are asked to:

\begin{enumerate}
    \item Derive a physical model highlighting assumptions and simplifications.

    \item Reproduce the acausal model in Dymola.

    \item Tune the cooling system in terms of volume flow rate, control logics, and initial fluid storage temperature such that:
    \begin{enumerate}
    
        \item the gear unit is kept between 40$^{\circ}$C and 60$^{\circ}$C.
        
        \item the source tank does not get empty before the end simulation time

        \item the storage tanks have a maximum height of 0.8 $m$ and cross section area of 0.01 $m^2$
        
        \item the system shall have a recirculating capability in order to exploit the outlet fluid for a next cooling process (when the source tank get empty)
        
        \item the sink heated fluid is kept between 5$^{\circ}$C and 10$^{\circ}$C.

        \item the power consumption of the thermal system shall be no more than 6 kW 
        
    \end{enumerate}

    \item Discuss the simulation results and the integration scheme used
     
    
\end{enumerate}

\noindent For the simulation part consider properties of water at 10$^{\circ}$C as cooling incompressible fluid (convective thermal conductance $\lambda_{conv}$ = 300 W/K) and the cylindrical pipe line data listed in Table \ref{tab:pipe}. The simulation shall last at least $\mathbf{t_{sim}}$ = 300 s starting with no water along the pipe.


\begin{table}[h!]
    \centering
    \caption{Pipe line properties}
\begin{tabular}{ |c|c|c| } 
\hline
\textbf{Parameter} & \textbf{Value} & \textbf{Unit}\\
\hline
Diameter & 4 & cm  \\ 
Length & 40 & cm  \\ 
Geodetic height & 0 & m \\
Friction losses & 0 & [-] \\ 
\hline
\end{tabular}
\label{tab:pipe}
    \end{table}

    
\medskip


\medskip \hrule \medskip
\rightline{\small(15 points)}

\tr{\textit{Write your answer here}}



\clearpage



\end{document}
%--------------------------------------------------------------------------------
%       END DOCUMENT
%--------------------------------------------------------------------------------