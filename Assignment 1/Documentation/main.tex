
\documentclass[11pt,a4paper,oneside]{article}
\usepackage[T1]{fontenc} 
\usepackage[utf8]{inputenc}
\usepackage[main=english]{babel}
\usepackage{graphicx} % 1pt = 0.035146cm
\graphicspath{{Figures/}}
\usepackage[justification=default]{subfig} %Manage sub-figures 
\usepackage[update]{epstopdf}
\usepackage[labelfont=bf]{caption}
\usepackage{titlesec} % Allows customization of titles
\usepackage{booktabs}

\usepackage{color}
\usepackage[textwidth = 450pt,top = 80pt, bottom = 60pt]{geometry}
\usepackage{soul}
\newcommand{\hlc}[2][yellow]{{\sethlcolor{#1}\hl{#2}}}
\usepackage[inline]{enumitem}
\usepackage[symbol]{footmisc}

\renewcommand{\thefootnote}{\fnsymbol{footnote}}

%--------------------------------------------------------------------------------
%       MATH PACKAGES
%--------------------------------------------------------------------------------
\usepackage{amsmath}
%\usepackage{mathtools}
\usepackage[leqno,fleqn,intlimits]{empheq}
\usepackage{bm}
%\usepackage{amssymb}
\usepackage{empheq}

%--------------------------------------------------------------------------------
%       MATLAB CODE
%--------------------------------------------------------------------------------
\usepackage{mcode}

%--------------------------------------------------------------------------------
%       MY COMMANDS
%--------------------------------------------------------------------------------
\renewcommand{\vec}[1]{\mathbf{#1}}
\newcommand{\tr}{\textcolor{red}}
\newcommand{\mathbi}[1]{\bm{\textbf{\em #1}}}

%--------------------------------------------------------------------------------
%       BIBLIOGRAPHY PACKAGES
%--------------------------------------------------------------------------------
% \usepackage{csquotes}
% \usepackage[sorting=nyt,%
% sortcites=true,%
% bibencoding=ascii,%
% autopunct=true,%
% hyperref=true,%
% language=auto,%
% %backref=true,%
% url=false,%
% maxcitenames=10,%
% minbibnames = 3,%
% maxbibnames=3,%
% giveninits, 
% natbib = false,
% isbn=false,%
% backend=biber]{biblatex}
% \addbibresource{bibliograhy_assignment.tex}

%--------------------------------------------------------------------------------
%       MISCELLANEA
%--------------------------------------------------------------------------------
\usepackage[]{hyperref}
\usepackage{cleveref}
%%% CREF setup
\crefname{equation}{Eq.}{Eqs.}
\crefname{table}{Table}{Tables}
\crefname{figure}{Fig.}{Figs.}

%--------------------------------------------------------------------------------
%       TITLE SECTION
%--------------------------------------------------------------------------------
\newcommand\headlinecolor{\normalcolor}

\makeatletter
\renewcommand*\maketitle{
    \begingroup
    \centering
    \fontsize{15}{15}% 72pt on 80pt leading
    \selectfont
    \headlinecolor
    \@title\\
    \vspace{5mm}
    \@author
    \par
    \vskip1in
    \endgroup
    \vspace{-22mm}
}
\makeatother

\title{MSAS -- Assignment \#1: Simulation} % Article title
\author{\large Matteo Baio, 10667431}
\date{}

%--------------------------------------------------------------------------------
% HEADING packages
\usepackage{fancyhdr}   % Headers and footers control
\setlength{\headheight}{32pt}
\pagestyle{fancyplain}  % Defines a new header for all pages (absolutely all pages, use fancy to exclude title-page and chapters, if book class is used) 
\fancyhf{}              % clears the header and footer, otherwise the elements of the default "plain" page style will appear
\lhead{Matteo Baio, MSAS -- Assignment \#1}
\rhead{\vspace{-0.5cm}\includegraphics[width=0.3\textwidth]{figure/newlogo.eps}}
\lfoot{AY 2023-24 -- Prof.\ F.\ Topputo; TA: C.\ Balossi, S.\ Borgia}
\rfoot{\thepage}


%--------------------------------------------------------------------------------
%       BEGIN DOCUMENT
%--------------------------------------------------------------------------------
\begin{document}

\maketitle

\thispagestyle{fancy}

\section{Implicit equations}

\subsection{Exercise 1}

Let $\vec{f}$ be a two-dimensional vector-valued function $\vec{f}(\vec{x}) = (x_2^2-x_1-2, \ -x_1^2+x_2+10)^\top$, where $\vec{x} = (x_1, x_2)^\top$. Find the zero(s) of $\vec{f}$ by using Newton's method with $\partial\vec f/\partial\vec x$ 
\begin{enumerate*}[label=\arabic*)]
    \item computed analytically, and
    \item estimated through finite differences.
\end{enumerate*}
Which version is more accurate?



\rightline{\small(3 points)}
\medskip    \hrule  \medskip

\noindent 
Prova

\includegraphics*[width=1\textwidth, height=10cm, keepaspectratio]{figure/ex1_initGuess.eps}

%\tr{\textit{Write your answer here}}


%%%%%%%%%% TO BE DELETED (BEGINS) %%%%%%%%%%
\tr{
\begin{itemize}
\item Develop the exercises in \underline{one Matlab script}; name the file \mcode{lastname123456_Assign1.m}
\item Organize the script in sections, one for each exercise; use local functions if needed. 
\item Download the \underline{PDF} from the Main menu.
\item \underline{Create a single .zip file} containing both the \underline{report in PDF} and the \underline{MATLAB file}. The name shall be \mcode{lastname123456_Assign1.zip}.
\item Red text indicates where answers are needed; be sure there is no red stuff in your report.
\item In your answers, \underline{be concise}: to the point.
\item \textbf{Deadline for the submission: Nov 20 2023, 23:59}.
\item \textbf{Load the compressed file to the Homework folder on Webeep}.
\end{itemize}
}
%%%%%%%%%% TO BE DELETED (ENDS) %%%%%%%%%%

%--------------------------------------------------------------------------------
%       IMPLICIT EQUATION
%--------------------------------------------------------------------------------




\clearpage

%--------------------------------------------------------------------------------
%       NUMERICAL SOLUTION OF NONLINEAR ODE
%--------------------------------------------------------------------------------

\section{Numerical solution of ODE}
\subsection*{Exercise 2}

The Initial Value Problem $\dot x = x- 2t^2+2$, $x(0) = 1$, has analytic solution $x(t) = 2t^2 + 4t - e^t + 2$. 
\begin{enumerate*}[label=\arabic*)]
    \item Implement a general-purpose, fixed-step Heun's method (RK2);
    \item Solve the IVP in $t\in[0,2]$ for $h_1 = 0.5$, $h_2 = 0.2$, $h_3 = 0.05$, $h_4 = 0.01$ and compare the numerical vs the analytical solution;
    \item Repeat points 1)--2) with RK4;
    \item Trade off between CPU time \& integration error.
\end{enumerate*}

\rightline{\small(4 points)}
\medskip \hrule \medskip

\tr{\textit{Write your answer here}}

\subsection*{Exercise 3}

Let $\dot{\vec x} = A(\alpha) \vec x$ be a two-dimensional system with $A(\alpha) = [0, 1; -1, 2\cos\alpha]$. Notice that $A(\alpha)$ has a pair of complex conjugate eigenvalues on the unit circle; $\alpha$ denotes the angle from the $\operatorname{Re}\{\lambda\}$-axis. 
\begin{enumerate*}[label=\arabic*)]
    \item Write the operator $F_{\rm RK2}(h,\alpha)$ that maps $\vec x_k$ into $\vec x_{k+1}$, namely $\vec x_{k+1} = F_{\rm RK2}(h,\alpha) \, \vec x_k$.
    \item\!With $\alpha = \pi$, solve the problem ``Find $h\ge 0$ s.t.$\max\left(|{\rm eig}(F(h,\alpha))|\right) = 1$''.
    \item Repeat point 2) for $\alpha\in[0, \pi]$ and draw the solutions in the $(h\lambda)$-plane.
    \item Repeat points 1)--3) with RK4.
\end{enumerate*}

\rightline{\small(5 points)}
\medskip \hrule \medskip

\tr{\textit{Write your answer here}}


\subsection*{Exercise 4}

Consider the IVP $\dot{\vec x}=A(\alpha)\vec x$, $\vec x(0) = [1, 1]^T$, to be integrated in $t\in[0, 1]$.
\begin{enumerate*}[label=\arabic*)]
    \item Take $\alpha\in[0, \pi]$ and solve the problem ``Find $h\ge 0$ s.t. $\left\|\vec x_{\rm an}(1)-\vec x_{\rm RK1}(1)\right\|_\infty = \mathrm{tol}$'', where $\vec x_{\rm an}(1)$ and $\vec x_{\rm RK1}(1)$ are the analytical and the numerical solution (with RK1) at the final time, respectively, and $\rm tol = \{10^{-3}, 10^{-4}, 10^{-5}, 10^{-6}\}$.
    \item Plot the four locus of solutions in the $(h\lambda)$-plane; plot also the function evaluations vs tol for $\alpha= \pi$.
    \item Repeat points 1)--2) for RK2 and RK4.
\end{enumerate*}

\rightline{\small(4 points)}
\medskip \hrule \medskip

\tr{\textit{Write your answer here}}

\subsection*{Exercise 5}
Consider the backinterpolation method $\textrm{BI2}_{0.4}$. 1) Derive the expression of the linear operator $B_{\rm BI2_{0.4}}(h,\alpha)$ such that $\vec x_{k+1} = B_{\rm BI2_{0.4}}(h,\alpha) \vec x_k$. 2) Following the approach of point 3) in Exercise 3, draw the stability domain of $\textrm{BI2}_{0.4}$ in the $(h\lambda)$-plane. 3) Derive the domain of numerical stability of $\textrm{BI2}_{\theta}$ for the values of $\theta = [0.1,\, 0.3,\, 0.7,\, 0.9]$.

\rightline{\small(5 points)}
\medskip \hrule \medskip

\tr{\textit{Write your answer here}}

\subsection*{Exercise 6}

Consider the IVP $\dot{\vec x} = B \vec x$ with $B =[-180.5, 219.5; 179.5, -220.5]$ and $\vec{x}(0)=[1, 1]^T$ to be integrated in $t\in[0, 5]$. Notice that $\vec{x}(t)=e^{Bt}\vec{x}(0)$.
\begin{enumerate*}[label=\arabic*)]
    \item Solve the IVP using RK4 with $h=0.1$;
    \item Repeat point 1) using implicit extrapolation technique IEX4;
    \item Compare the numerical results in points 1) and 2) against the analytic solution;
    \item Compute the eigenvalues associated to the IVP and represent them on the $(h\lambda)$-plane both for RK4 and IEX4;
    \item Discuss the results.
\end{enumerate*}

\rightline{\small(4 points)}
\medskip \hrule \medskip

\tr{\textit{Write your answer here}}


\subsection*{Exercise 7}

Consider the two-dimensional IVP 
$$\begin{bmatrix}\dot{x}_1 \\ \dot{x}_2\end{bmatrix}=\begin{bmatrix}-\frac{5}{2}\left[1+8\sin(t)\right]x_1 \\ (1-x_1)x_2+x_1\end{bmatrix}, \qquad \begin{bmatrix} x_1(t_0)\\ x_2(t_0)\end{bmatrix}=\begin{bmatrix} 1\\ 1\end{bmatrix}$$
\begin{enumerate*}[label=\arabic*)]
    \item Solve the IVP using AB3 in $t\in[0,3]$ for $h=0.1$;
    \item Repeat point 1) using AM3, ABM3, and BDF3;
    \item Discuss the results.
\end{enumerate*}

\rightline{\small(5 points)}
\medskip \hrule \medskip



\tr{\textit{Write your answer here}}



\end{document}

%--------------------------------------------------------------------------------
%       END DOCUMENT
%--------------------------------------------------------------------------------